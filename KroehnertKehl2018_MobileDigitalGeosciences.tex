\documentclass[review]{elsarticle}

\usepackage{lineno}
\modulolinenumbers[5]

\usepackage[UKenglish]{babel}
\usepackage[reqno,fleqn]{amsmath}	% erweiteter Formelsatz und zus�tzliche Mathe-Symbole
\usepackage{breqn}
\usepackage{amssymb}
\usepackage{amsfonts}
\usepackage[mathcal]{euscript} % For caligraphy fonts
\usepackage{dsfont}
\usepackage{makeidx}
\usepackage{tabularx}
\usepackage{longtable}

\usepackage{graphicx}
%\usepackage[pdftex]{graphicx}
% declare the path(s) where your graphic files are
%\graphicspath{{.},{./publishImages/}}
\graphicspath{{.},{./images/}}
\DeclareGraphicsExtensions{.pdf,.jpeg,.png,.tif,.jpg}
\usepackage[center,tight,footnotesize]{subfigure}

%\usepackage{lipsum}
%\usepackage{xcolor,colortbl,moreverb}
%\usepackage{color}
\usepackage[dvipsnames]{xcolor}
%\usepackage{shortvrb}
%\usepackage{url}
\usepackage[hidelinks]{hyperref}
\usepackage{booktabs}
%\usepackage[hidelinks,bookmarks=true]{hyperref}
%\usepackage{bookmark}

%\usepackage[pdftex,
%			backref,         % List citing occurences in the References
%			colorlinks,      % Colored links
%			citecolor=black,  % Color of cite links
%			linkcolor=black,  % Color of links
%			urlcolor=blue   % Color of urls
%			]{hyperref}

%\usepackage{ctable}
%\usepackage{gensymb}
%\usepackage{textcomp}
\usepackage{pdfpages}
%\usepackage{cite}
\usepackage{footnote}

\usepackage{lscape} % or {pdflscape}
\usepackage{longtable}
\usepackage{multirow}
\usepackage{color, colortbl}

%\definecolor{LinkColor}{rgb}{0,0,0.5}
%\definecolor{orange}{rgb}{1.0,0.5,0}
%\definecolor{ORANGE}{RGB}{255,165,0}
%\definecolor{green}{rgb}{0,0.61,0.33}
%\definecolor{blue}{rgb}{0,0.3,0.65}
%\definecolor{red}{RGB}{255,99,71}

%\usepackage{xargs}
%\newcommandx{\chris}[1]{{\color{blue} \textit{(C)} #1}}
%\newcommandx{\sophie}[1]{{\color{purple} \textit{(J)} #1}}
%\newcommandx{\change}[1]{{\color{orange} #1}}
%\newcommandx{\strike}[1]{\change{\sout{#1}}}

%\newcommandx{\chris}[1]{#1}
%\newcommandx{\sophie}[1]{#1}
%\newcommandx{\change}[1]{#1}
%\newcommandx{\strike}[1]{}

%\hyphenpenalty=0
\usepackage[acronyms,shortcuts]{glossaries}
\renewcommand*{\acronymfont}[1]{\mbox{#1}}
\hyphenpenalty=1
\tolerance=1000
%\let\oldnewacronym\newacronym
%\renewcommand\newacronym[3]{\hyphenation{#2}\oldnewacronym{#1}{#2}{#3}}

\hyphenation{Li-dar Map-ping and In-ter-pre-ta-tion En-vi-ron-ment}
\hyphenation{Dis-crete}
\hyphenation{In-ter-po-la-tion}
\hyphenation{di-gi-tal}
\hyphenation{ele-va-tion}
\hyphenation{mo-del}
\makeglossaries

\newacronym{CT}{CT}{computed tomography}
\newacronym{MRI}{MRI}{Magnet Resonance Imaging}
\newacronym{DTI}{DTI}{Diffusion Tensor Imaging}
\newacronym{DSI}{DSI}{Dis\-crete Smooth In\-ter\-po\-la\-tion}
\newacronym{CAD}{CAD}{computer-aided design}
\newacronym{CFD}{CFD}{computational fluid dynamics}
\newacronym{DEM}{DEM}{digital elevation model}
\newacronym{DSM}{DSM}{digital surface model}
\newacronym{DTM}{DTM}{digital terrain model}
%\newacronym{LiDAR}{LiDAR}{light detection and range}
\newacronym{LiDAR}{lidar}{light detection and range}
\newacronym{VOM}{VOM}{virtual outcrop model}
\newacronym{DOM}{DOM}{digital outcrop model}
\newacronym{FDM}{FDM}{facies distribution map}
\newacronym{GRIT}{GRIT}{Geological Registration and Interpretation Toolset}
\newacronym{LIME}{LIME}{Lidar Interpretation Mapping Environment}
\newacronym{VRGS}{VRGS}{Virtual Reality Geological Studio}
\newacronym[\glsshortpluralkey={LoD's},\glslongpluralkey={Levels-of-Detail}]{LoD}{LoD}{Level-of-Detail}
\newacronym{KML}{KML}{Keyhole Markup Language}
\newacronym{PID}{PID}{Proportional-Integral-Differential}
\newacronym{PBR}{PBR}{Point-based Rendering}
\newacronym{SVM}{SVM}{Support Vector Machine}
\newacronym{RLE}{RLE}{Runlength Encoding}
\newacronym{VDB}{VDB}{Volumetric Dynamic Grid B+Tree}
\newacronym[\glsshortpluralkey={LoA's},\glslongpluralkey={Levels-of-Abstraction}]{LoA}{LoA}{Level-of-Abstraction}
\newacronym[\glsshortpluralkey={GPUs},\glslongpluralkey={graphics processing units}]{GPU}{GPU}{graphics processing unit}
\newacronym{CPU}{CPU}{central processing unit}
\newacronym[\glsshortpluralkey={SDIs},\glslongpluralkey={spatial data infrastructures}]{SDI}{SDI}{spatial data infrastructure}
\newacronym[\glsshortpluralkey={TINs},\glslongpluralkey={triangulated irregular networks}]{TIN}{TIN}{triangulated irregular network}
\newacronym{GML}{GML}{Geography Markup Language}
\newacronym{XML}{XML}{Extensible Markup Language}
\newacronym{VRML}{VRML}{Virtual Reality Markup Language}
\newacronym[\glsshortpluralkey={GIS},\glslongpluralkey={geographic information systems}]{GIS}{GIS}{geographic information system}
\newacronym{OGR}{OGR}{OGR Simple Features Library}
\newacronym{GDAL}{GDAL}{Geospatial Data Abstraction Library}
\newacronym{GNSS}{GNSS}{global navigation satellite system}
\newacronym{GPS}{GPS}{global positioning system}
\newacronym{dGPS}{dGPS}{differential GPS}
\newacronym{OSM}{OSM}{Open Street Map}
\newacronym{SLR}{SLR}{single-lens reflex}
\newacronym{DSLR}{DSLR}{digital single lens reflex}
\newacronym{SBA}{SBA}{Sparse Bundle Adjustment}
\newacronym{MPS}{MPS}{multiple point statistics}
\newacronym{DLT}{DLT}{Direct Linear Transform}
\newacronym{MPCD}{MPCD}{Mobile Personal Communication Device}
\newacronym{MI}{MI}{Mutual Information}
\newacronym{SLAM}{SLAM}{simultaneous localisation and mapping}
\newacronym{SIFT}{SIFT}{Scale-Invariant Feature Transform}
\newacronym{SURF}{SURF}{Speeded-Up Robust Features}
\newacronym{MSER}{MSER}{Maximally Stable Extremal Regions}
\newacronym{MSCR}{MSCR}{Maximally Stable Colour Regions}
\newacronym{SfM}{SfM}{structure from motion}
\newacronym{RANSAC}{RANSAC}{Random Sampling Consensus}
%\newacronym{EPnP}{EPnP}{Efficient Perspective-n-Point}
\newacronym{EPnP}{EPnP}{Efficient PnP}
\newacronym{ICP}{ICP}{Iterative Closest Point}
\newacronym{VGI}{VGI}{Volunteered Geographic Information}
\newacronym{UAV}{UAV}{unmanned aerial vehicle}
\newacronym{TLS}{TLS}{terrestrial laser scanning}
\newacronym{ToF}{ToF}{time-of-flight}
\newacronym{TI}{TI}{training image}
\newacronym{LM}{LM}{Levenberg-Marquardt}
\newacronym{PnP}{PnP}{Point-n-Perspective}
\newacronym{AR}{AR}{augmented reality}
\newacronym{VR}{VR}{virtual reality}
%\newacronym{PLS}{PLS}{piecewise-linear simplex}
\newacronym[\glsshortpluralkey={PLSs},longplural={piecewise-linear simplices}]{PLS}{PLS}{piecewise-linear simplex}
%\newacronym{PLC}{PLC}{piecewise-linear complex}
\newacronym[\glsshortpluralkey={PLCs},longplural={piecewise-linear complices}]{PLC}{PLC}{piecewise-linear complex}
\newacronym{CG}{CG}{computer graphics}
\newacronym{CGI}{CGI}{computer-generated imagery}
\newacronym{CV}{CV}{computer vision}
\newacronym{CDT}{CDT}{constrained Delaunay triangulation}
\newacronym{FEA}{FEA}{finite-element analysis}
\newacronym{CGAL}{CGAL}{Computational Geometry Algorithms Library}
\newacronym{THMC}{THMC}{thermal, hydraulic, mechanical and chemical}
\newacronym{DCT}{DCT}{discrete cosine transform}
\newacronym{PSS}{PSS}{point set surface}
\newacronym{WYSIWYG}{WYSIWYG}{what-you-see-is-what-you-get}
\newacronym{MLS}{MLS}{moving least squares}
\newacronym{SSE}{SSE}{streaming SIMD extensions}
\newacronym{GLES}{GLES}{graphics library for embedded systems}
\newacronym{CEREGE}{CEREGE}{Centre Europ\'{e}en de Recherche et d'Enseignement des G\'{e}osciences de l'Environnement}
\newacronym[\glsshortpluralkey={IMUs},longplural={initial measurement units}]{IMU}{IMU}{initial measurement unit}
\newacronym[\glsshortpluralkey={INSs},longplural={initial navigation systems}]{INS}{INS}{initial navigation system}
\newacronym[\glsshortpluralkey={RMSEs},longplural={root mean square errors}]{RMSE}{RMSE}{root mean square error}

%
%\newglossaryentry{CT}
%{
%	type=\acronymtype, 
%	name={CT}, 
%	description={computer tomography}, 
%	text={CT}, 
%	first={computer tomography (CT)},
%}

%\newglossaryentry{MRI}
%{ 
%	type=\acronymtype, 
%	name={MRI}, 
%	description={magnet resonance imaging}, 
%	text={MRI}, 
%	first={magnet resonance imaging (MRI)},
%}

\journal{Computers \& Geosciences}

%%%%%%%%%%%%%%%%%%%%%%%
%% Elsevier bibliography styles
%%%%%%%%%%%%%%%%%%%%%%%
%% To change the style, put a % in front of the second line of the current style and
%% remove the % from the second line of the style you would like to use.
%%%%%%%%%%%%%%%%%%%%%%%

%% Numbered
\bibliographystyle{model1-num-names}

%% Numbered without titles
%\bibliographystyle{model1a-num-names}

%% Harvard
%\bibliographystyle{model2-names.bst}\biboptions{authoryear}

%% Vancouver numbered
%\usepackage{numcompress}\bibliographystyle{model3-num-names}

%% Vancouver name/year
%\usepackage{numcompress}\bibliographystyle{model4-names}\biboptions{authoryear}

%% APA style
%\bibliographystyle{model5-names}\biboptions{authoryear}

%% AMA style
%\usepackage{numcompress}\bibliographystyle{model6-num-names}

%% `Elsevier LaTeX' style
%\bibliographystyle{elsarticle-num}
%%%%%%%%%%%%%%%%%%%%%%%

\begin{document}\setlength\emergencystretch{1.5em}

\begin{frontmatter}

%\title{Elsevier \LaTeX\ template\tnoteref{mytitlenote}}
\title{Digital Geosciences on Mobile Devices - Concepts, Challenges and Applications}
%\tnotetext[mytitlenote]{Fully documented templates are available in the elsarticle package on \href{http://www.ctan.org/tex-archive/macros/latex/contrib/elsarticle}{CTAN}.}

%% Group authors per affiliation:
%\author{Elsevier\fnref{myfootnote}}
%\address{Radarweg 29, Amsterdam}
%\fntext[myfootnote]{Since 1880.}
%\ead[url]{www.elsevier.com}

%% or include affiliations in footnotes:
%\author[anonymous]{Anonymous\corref{correspondence}}
%\cortext[correspondence]{Corresponding author}
%\ead{anonymous}

\author[tudresden]{Melanie Kr\"{o}hnert\corref{correspondence}}
\cortext[correspondence]{Corresponding author}
\ead{melanie.kroehnert@tu-dresden.de}

\author[cerege,dtu]{Christian Kehl}
\ead{chke@dtu.dk}


\author[cerege]{Sophie Viseur}
\ead{viseur@cerege.fr}

\author[uniresearch,uib]{Simon J. Buckley}
\ead{Simon.Buckley@uni.no}

%\address[anonymous]{anonymous}
\address[tudresden]{Institute for Photogrammetry \& Remote Sensing, TU Dresden, Helmholtzstr. 10, 01069 Dresden, Germany}
\address[cerege]{Aix Marseille Universit\'{e}, CNRS, IRD, CEREGE UM 34, Dept. Sedimentary and Reservoir Systems, 13001 Marseille, France}
\address[uniresearch]{Uni Research AS -- CIPR, Nyg{\aa}rdsgaten 112, 5008 Bergen, Norway}
\address[uib]{Department of Earth Science, University of Bergen, All\'{e}gaten 41, 5007 Bergen, Norway}
\address[dtu]{}


\begin{abstract}

\end{abstract}

\textbf{Highlights} \\
\begin{itemize}
\item point 1
\item point 2
\item point 3
\item point 4
\item point 5
\end{itemize}

\begin{keyword}
discrete geometry\sep surface reconstruction\sep volume reconstruction\sep surface parameterization\sep digital outcrops
\MSC[2010] 00-01\sep  99-00
\end{keyword}

\end{frontmatter}

\linenumbers

\section{Introduction}
\label{sec:introduction}

\begin{itemize}
\item computing equipment continuously elevates the analytical capabilities for solving geoscientific problems
\item large drawback on computing equipment: the more powerful it is, the more stationary it is
\item geoscience disciplines such as hydrology, geology or glaciology are driven by outdoor experiments that prohibit bulky equipment
\item the advent of mobile computing equipment, such as smartphones and tablets, provides a possible solution to the equipment problem
\item form factor of mobile devices is small enough to allow every field-related geoscientist to carry one in the field
\item as seen is popular articles, the range of available devices increases, which allows to find a devices fit-for-purpose to each situation
\item range of devices also comes with a range of capabilities that influence their usability for specific field tasks
\end{itemize}

\begin{itemize}
\item availability of small form factor devices is only on part contribution to making digital geosciences more ''mobile``
\item availability and easy access to geoscience data (e.g. domain-specific maps, \glspl{DEM}, surface models in 3D) is equally important to perform combined digital- and field analysis
\item while basemap access on mobile devices is trivial, surface-scanned data in form of point clouds and (textured) triangulated meshes is becoming increasingly available with novice-operable \gls{SfM} software and drones
\item crowdsourced data and \gls{VGI} provides numerous data for domain-specific analysis, which is facilitated by easier data capture from amateur scientists using mobile devices
\end{itemize}

\begin{itemize}
\item In order to connect data and devices in the field, domain-specific mobile software is required
\item the difficulties in mobile software development stem from the specific demands and challenges for mobile software, such as energy efficiency, multi-manufacturer support, smart sensor utilisation [add and expand]
\item with the emergence of new application cases, which are demonstrated and discussed in this article, and an increasing interest from geoscience- and computer technology industry, a significant rise in the mobile software availability for geoscience problem solving is expected for the near-term future
\end{itemize}

\begin{itemize}
\item Challenges
\end{itemize}

\section{Target case studies}
\label{sec:case_studies}

\section{Representation basis -- Geometry and Radiometry}
\label{sec:representations}

\section{Algorithms}
\label{sec:algorithms}

\subsection{Structure-from-Motion model generation}

\subsection{Image-to-geometry}

\subsection{Data representation and rendering}

\subsection{Interpretation and annotation}

\section{Technology}
\label{sec:technology}

\section{Applications and Requirements}
\label{sec:applications}

\section{Conclusions}
\label{sec:conclusions}

\section{Discussion}
\label{sec:discussion}

\section*{References}

\bibliography{KroehnertKehl2018_MobileDigitalGeosciences}

\end{document}