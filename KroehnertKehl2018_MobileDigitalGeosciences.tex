\documentclass[review]{elsarticle}

\usepackage{lineno}
\modulolinenumbers[5]

\usepackage[UKenglish]{babel}
\usepackage[reqno,fleqn]{amsmath}	% erweiteter Formelsatz und zus�tzliche Mathe-Symbole
\usepackage{breqn}
\usepackage{amssymb}
\usepackage{amsfonts}
\usepackage[mathcal]{euscript} % For caligraphy fonts
\usepackage{dsfont}
\usepackage{makeidx}
\usepackage{tabularx}
\usepackage{longtable}

\usepackage{graphicx}
%\usepackage[pdftex]{graphicx}
% declare the path(s) where your graphic files are
%\graphicspath{{.},{./publishImages/}}
\graphicspath{{.},{./images/}}
\DeclareGraphicsExtensions{.pdf,.jpeg,.png,.tif,.jpg}
\usepackage[center,tight,footnotesize]{subfigure}

%\usepackage{lipsum}
%\usepackage{xcolor,colortbl,moreverb}
%\usepackage{color}
\usepackage[dvipsnames]{xcolor}
%\usepackage{shortvrb}
%\usepackage{url}
\usepackage[hidelinks]{hyperref}
\usepackage{booktabs}
%\usepackage[hidelinks,bookmarks=true]{hyperref}
%\usepackage{bookmark}

%\usepackage[pdftex,
%			backref,         % List citing occurences in the References
%			colorlinks,      % Colored links
%			citecolor=black,  % Color of cite links
%			linkcolor=black,  % Color of links
%			urlcolor=blue   % Color of urls
%			]{hyperref}

%\usepackage{ctable}
%\usepackage{gensymb}
%\usepackage{textcomp}
\usepackage{pdfpages}
%\usepackage{cite}
\usepackage{footnote}

\usepackage{lscape} % or {pdflscape}
\usepackage{longtable}
\usepackage{multirow}
\usepackage{color, colortbl}

%\definecolor{LinkColor}{rgb}{0,0,0.5}
%\definecolor{orange}{rgb}{1.0,0.5,0}
%\definecolor{ORANGE}{RGB}{255,165,0}
%\definecolor{green}{rgb}{0,0.61,0.33}
%\definecolor{blue}{rgb}{0,0.3,0.65}
%\definecolor{red}{RGB}{255,99,71}

%\usepackage{xargs}
%\newcommandx{\chris}[1]{{\color{blue} \textit{(C)} #1}}
%\newcommandx{\sophie}[1]{{\color{purple} \textit{(J)} #1}}
%\newcommandx{\change}[1]{{\color{orange} #1}}
%\newcommandx{\strike}[1]{\change{\sout{#1}}}

%\newcommandx{\chris}[1]{#1}
%\newcommandx{\sophie}[1]{#1}
%\newcommandx{\change}[1]{#1}
%\newcommandx{\strike}[1]{}

%\hyphenpenalty=0
\usepackage[acronyms,shortcuts]{glossaries}
\renewcommand*{\acronymfont}[1]{\mbox{#1}}
\hyphenpenalty=1
\tolerance=1000
%\let\oldnewacronym\newacronym
%\renewcommand\newacronym[3]{\hyphenation{#2}\oldnewacronym{#1}{#2}{#3}}

\hyphenation{Li-dar Map-ping and In-ter-pre-ta-tion En-vi-ron-ment}
\hyphenation{Dis-crete}
\hyphenation{In-ter-po-la-tion}
\hyphenation{di-gi-tal}
\hyphenation{ele-va-tion}
\hyphenation{mo-del}
\makeglossaries

\newacronym{CT}{CT}{computed tomography}
\newacronym{MRI}{MRI}{Magnet Resonance Imaging}
\newacronym{DTI}{DTI}{Diffusion Tensor Imaging}
\newacronym{DSI}{DSI}{Dis\-crete Smooth In\-ter\-po\-la\-tion}
\newacronym{CAD}{CAD}{computer-aided design}
\newacronym{CFD}{CFD}{computational fluid dynamics}
\newacronym{DEM}{DEM}{digital elevation model}
\newacronym{DSM}{DSM}{digital surface model}
\newacronym{DTM}{DTM}{digital terrain model}
%\newacronym{LiDAR}{LiDAR}{light detection and range}
\newacronym{LiDAR}{lidar}{light detection and range}
\newacronym{VOM}{VOM}{virtual outcrop model}
\newacronym{DOM}{DOM}{digital outcrop model}
\newacronym{FDM}{FDM}{facies distribution map}
\newacronym{GRIT}{GRIT}{Geological Registration and Interpretation Toolset}
\newacronym{LIME}{LIME}{Lidar Interpretation Mapping Environment}
\newacronym{VRGS}{VRGS}{Virtual Reality Geological Studio}
\newacronym[\glsshortpluralkey={LoD's},\glslongpluralkey={Levels-of-Detail}]{LoD}{LoD}{Level-of-Detail}
\newacronym{KML}{KML}{Keyhole Markup Language}
\newacronym{PID}{PID}{Proportional-Integral-Differential}
\newacronym{PBR}{PBR}{Point-based Rendering}
\newacronym{SVM}{SVM}{Support Vector Machine}
\newacronym{RLE}{RLE}{Runlength Encoding}
\newacronym{VDB}{VDB}{Volumetric Dynamic Grid B+Tree}
\newacronym[\glsshortpluralkey={LoA's},\glslongpluralkey={Levels-of-Abstraction}]{LoA}{LoA}{Level-of-Abstraction}
\newacronym[\glsshortpluralkey={GPUs},\glslongpluralkey={graphics processing units}]{GPU}{GPU}{graphics processing unit}
\newacronym{CPU}{CPU}{central processing unit}
\newacronym[\glsshortpluralkey={SDIs},\glslongpluralkey={spatial data infrastructures}]{SDI}{SDI}{spatial data infrastructure}
\newacronym[\glsshortpluralkey={TINs},\glslongpluralkey={triangulated irregular networks}]{TIN}{TIN}{triangulated irregular network}
\newacronym{GML}{GML}{Geography Markup Language}
\newacronym{XML}{XML}{Extensible Markup Language}
\newacronym{VRML}{VRML}{Virtual Reality Markup Language}
\newacronym[\glsshortpluralkey={GIS},\glslongpluralkey={geographic information systems}]{GIS}{GIS}{geographic information system}
\newacronym{OGR}{OGR}{OGR Simple Features Library}
\newacronym{GDAL}{GDAL}{Geospatial Data Abstraction Library}
\newacronym{GNSS}{GNSS}{global navigation satellite system}
\newacronym{GPS}{GPS}{global positioning system}
\newacronym{dGPS}{dGPS}{differential GPS}
\newacronym{OSM}{OSM}{Open Street Map}
\newacronym{SLR}{SLR}{single-lens reflex}
\newacronym{DSLR}{DSLR}{digital single lens reflex}
\newacronym{SBA}{SBA}{Sparse Bundle Adjustment}
\newacronym{MPS}{MPS}{multiple point statistics}
\newacronym{DLT}{DLT}{Direct Linear Transform}
\newacronym{MPCD}{MPCD}{Mobile Personal Communication Device}
\newacronym{MI}{MI}{Mutual Information}
\newacronym{SLAM}{SLAM}{simultaneous localisation and mapping}
\newacronym{SIFT}{SIFT}{Scale-Invariant Feature Transform}
\newacronym{SURF}{SURF}{Speeded-Up Robust Features}
\newacronym{MSER}{MSER}{Maximally Stable Extremal Regions}
\newacronym{MSCR}{MSCR}{Maximally Stable Colour Regions}
\newacronym{SfM}{SfM}{structure from motion}
\newacronym{RANSAC}{RANSAC}{Random Sampling Consensus}
%\newacronym{EPnP}{EPnP}{Efficient Perspective-n-Point}
\newacronym{EPnP}{EPnP}{Efficient PnP}
\newacronym{ICP}{ICP}{Iterative Closest Point}
\newacronym{VGI}{VGI}{Volunteered Geographic Information}
\newacronym{UAV}{UAV}{unmanned aerial vehicle}
\newacronym{TLS}{TLS}{terrestrial laser scanning}
\newacronym{ToF}{ToF}{time-of-flight}
\newacronym{TI}{TI}{training image}
\newacronym{LM}{LM}{Levenberg-Marquardt}
\newacronym{PnP}{PnP}{Point-n-Perspective}
\newacronym{AR}{AR}{augmented reality}
\newacronym{VR}{VR}{virtual reality}
%\newacronym{PLS}{PLS}{piecewise-linear simplex}
\newacronym[\glsshortpluralkey={PLSs},longplural={piecewise-linear simplices}]{PLS}{PLS}{piecewise-linear simplex}
%\newacronym{PLC}{PLC}{piecewise-linear complex}
\newacronym[\glsshortpluralkey={PLCs},longplural={piecewise-linear complices}]{PLC}{PLC}{piecewise-linear complex}
\newacronym{CG}{CG}{computer graphics}
\newacronym{CGI}{CGI}{computer-generated imagery}
\newacronym{CV}{CV}{computer vision}
\newacronym{CDT}{CDT}{constrained Delaunay triangulation}
\newacronym{FEA}{FEA}{finite-element analysis}
\newacronym{CGAL}{CGAL}{Computational Geometry Algorithms Library}
\newacronym{THMC}{THMC}{thermal, hydraulic, mechanical and chemical}
\newacronym{DCT}{DCT}{discrete cosine transform}
\newacronym{PSS}{PSS}{point set surface}
\newacronym{WYSIWYG}{WYSIWYG}{what-you-see-is-what-you-get}
\newacronym{MLS}{MLS}{moving least squares}
\newacronym{SSE}{SSE}{streaming SIMD extensions}
\newacronym{GLES}{GLES}{graphics library for embedded systems}
\newacronym{CEREGE}{CEREGE}{Centre Europ\'{e}en de Recherche et d'Enseignement des G\'{e}osciences de l'Environnement}
\newacronym[\glsshortpluralkey={IMUs},longplural={initial measurement units}]{IMU}{IMU}{initial measurement unit}
\newacronym[\glsshortpluralkey={INSs},longplural={initial navigation systems}]{INS}{INS}{initial navigation system}
\newacronym[\glsshortpluralkey={RMSEs},longplural={root mean square errors}]{RMSE}{RMSE}{root mean square error}

%
%\newglossaryentry{CT}
%{
%	type=\acronymtype, 
%	name={CT}, 
%	description={computer tomography}, 
%	text={CT}, 
%	first={computer tomography (CT)},
%}

%\newglossaryentry{MRI}
%{ 
%	type=\acronymtype, 
%	name={MRI}, 
%	description={magnet resonance imaging}, 
%	text={MRI}, 
%	first={magnet resonance imaging (MRI)},
%}

\journal{Computers \& Geosciences}

%%%%%%%%%%%%%%%%%%%%%%%
%% Elsevier bibliography styles
%%%%%%%%%%%%%%%%%%%%%%%
%% To change the style, put a % in front of the second line of the current style and
%% remove the % from the second line of the style you would like to use.
%%%%%%%%%%%%%%%%%%%%%%%

%% Numbered
\bibliographystyle{model1-num-names}

%% Numbered without titles
%\bibliographystyle{model1a-num-names}

%% Harvard
%\bibliographystyle{model2-names.bst}\biboptions{authoryear}

%% Vancouver numbered
%\usepackage{numcompress}\bibliographystyle{model3-num-names}

%% Vancouver name/year
%\usepackage{numcompress}\bibliographystyle{model4-names}\biboptions{authoryear}

%% APA style
%\bibliographystyle{model5-names}\biboptions{authoryear}

%% AMA style
%\usepackage{numcompress}\bibliographystyle{model6-num-names}

%% `Elsevier LaTeX' style
%\bibliographystyle{elsarticle-num}
%%%%%%%%%%%%%%%%%%%%%%%

\begin{document}\setlength\emergencystretch{1.5em}

\begin{frontmatter}

\title{Interactive interpretation of 3D surfaces in field-based geosciences using mobile devices - concepts, challenges and applications}
%\title{Digital Geosciences on Mobile Devices - Concepts, Challenges and Applications}
%\tnotetext[mytitlenote]{Fully documented templates are available in the elsarticle package on \href{http://www.ctan.org/tex-archive/macros/latex/contrib/elsarticle}{CTAN}.}

%% Group authors per affiliation:
%\author{Elsevier\fnref{myfootnote}}
%\address{Radarweg 29, Amsterdam}
%\fntext[myfootnote]{Since 1880.}
%\ead[url]{www.elsevier.com}

%% or include affiliations in footnotes:
%\author[anonymous]{Anonymous\corref{correspondence}}
%\cortext[correspondence]{Corresponding author}
%\ead{anonymous}

\author[tudresden]{Melanie Kr\"{o}hnert\corref{correspondence}}
\cortext[correspondence]{Corresponding author}
\ead{melanie.kroehnert@tu-dresden.de}

\author[cerege,dtu]{Christian Kehl}
\ead{chke@dtu.dk}


\author[cerege]{Sophie Viseur}
\ead{viseur@cerege.fr}

\author[uniresearch,uib]{Simon J. Buckley}
\ead{Simon.Buckley@uni.no}

%\address[anonymous]{anonymous}
\address[tudresden]{Institute for Photogrammetry \& Remote Sensing, TU Dresden, Helmholtzstr. 10, 01069 Dresden, Germany}
\address[cerege]{Aix Marseille Universit\'{e}, CNRS, IRD, CEREGE UM 34, Dept. Sedimentary and Reservoir Systems, 13001 Marseille, France}
\address[uniresearch]{Uni Research AS -- CIPR, Nyg{\aa}rdsgaten 112, 5008 Bergen, Norway}
\address[uib]{Department of Earth Science, University of Bergen, All\'{e}gaten 41, 5007 Bergen, Norway}
\address[dtu]{}


\begin{abstract}

\end{abstract}

\textbf{Highlights} \\
\begin{itemize}
\item point 1
\item point 2
\item point 3
\item point 4
\item point 5
\end{itemize}

\begin{keyword}
discrete geometry\sep surface reconstruction\sep volume reconstruction\sep surface parameterization\sep digital outcrops
\MSC[2010] 00-01\sep  99-00
\end{keyword}

\end{frontmatter}

\linenumbers

\section{Introduction}
\label{sec:introduction}

\begin{itemize}
\item computing equipment continuously elevates the analytical capabilities for solving geoscientific problems
\item large drawback on computing equipment: the more powerful it is, the more stationary it is
\item geoscience disciplines such as hydrology, geology or glaciology are driven by outdoor experiments that prohibit bulky equipment
\item the advent of mobile computing equipment, such as smartphones and tablets, provides a possible solution to the equipment problem
\item form factor of mobile devices is small enough to allow every field-related geoscientist to carry one in the field
\item as seen is popular articles, the range of available devices increases, which allows to find a devices fit-for-purpose to each situation
\item range of devices also comes with a range of capabilities that influence their usability for specific field tasks
\end{itemize}

\begin{itemize}
\item availability of small form factor devices is only on part contribution to making digital geosciences more ''mobile``
\item availability and easy access to geoscience data (e.g. domain-specific maps, \glspl{DEM}, surface models in 3D) is equally important to perform combined digital- and field analysis
\item while basemap access on mobile devices is trivial, surface-scanned data in form of point clouds and (textured) triangulated meshes is becoming increasingly available with novice-operable \gls{SfM} software and drones
\item crowdsourced data and \gls{VGI} provides numerous data for domain-specific analysis, which is facilitated by easier data capture from amateur scientists using mobile devices
\end{itemize}

\begin{itemize}
\item In order to connect data and devices in the field, domain-specific mobile software is required
\item the difficulties in mobile software development stem from the specific demands and challenges for mobile software, such as energy efficiency, multi-manufacturer support, smart sensor utilisation [add and expand]
\item with the emergence of new application cases, which are demonstrated and discussed in this article, and an increasing interest from geoscience- and computer technology industry, a significant rise in the mobile software availability for geoscience problem solving is expected for the near-term future
\end{itemize}

\begin{itemize}
\item Challenges
\end{itemize}

\section{Target case studies}
\label{sec:case_studies}

\section{Representation basis -- Geometry and Radiometry}
\label{sec:representations}

\section{Algorithms}
\label{sec:algorithms}

\subsection{Structure-from-Motion model generation}

\subsection{Image-to-geometry}

\subsection{Data representation and rendering}

\subsection{Interpretation and annotation}



\section{Technology}
\label{sec:technology}

\subsection{Sensors}

\subsubsection{Localization}

\subsubsection{Orientation}

\begin{itemize}
\item stability IMU (see 3D-NO)
\item precision IMU
\end{itemize}

\subsubsection{Parameter sensitivity}

\subsection{Graphics}

\begin{itemize}
\item software- vs hardware renderer
\item web-rendering
\item rendering-on-device
\item hardware differences: speed, capability, CUDA
\end{itemize}

%\subsection{Component differences between devices}
%
%\begin{itemize}
%\item camera
%\end{itemize}

\subsection{Power consumption}



\section{Applications and Requirements}
\label{sec:applications}
Due to the increasing usability of mobile devices for in the field annotations, several use cases concerning geosciences has become apparent. In the following, two essential 

\subsection{Derivation of hydrological parameters: Water level gauging}
\label{sec:water_level_gauging_intro}
The last decade is characterized by a continued increase of globally devastating flash floods after heavy rainfalls. Even smallest creeks turned into hazardous streams causing flooding and landslides. Conventional gauging stations provide precise information about water levels measured over a short time period. State of the art techniques for administrative observation comprise water pressure sensors, floating gauges and conventional tide gauges. They are characterised by long-term stability and outdoor robustness providing accuries of several millimeters up to one centimeter \cite{Siedschlag2015}. Averaged over defined time intervals, it is advisable to remain caution regarding these accuracies may be too optimistic \cite{Horner2018} . 
 
Because of high costs in purchase and maintenance, gauging stations with complex sensing devices must be sparsely installed. A prime example here is the hydrological network in Saxony, Germany. Here, 184 gauging stations are installed for permanent observation on 154 of 259 rivers rising from small, medium and large catchments \cite{Saxon2018, Buettner2015}. Thus, around a third is not monitored neither during flood events when the most protection is required. Recently, commercial smartphone applications arose to enable crowd-sourcing based water level estimation for, among other things, such cases \cite{CrowdWaterApp2017a, Kisters2014}. But all of them have one thing in common: the water level is entered manually by engaged citizen scientists finding and photographing tide gauges close to rivers that makes - on the one hand a potential danger to themselves (f.e. by sudden landslides), and still limits on the other the approaches to open and visible gauges.

\subsection{Development of a versatile mobile water level application}
\label{sec:water_level_gauging_approach}
Improvements in this sense can be achieved through \textit{image-2-geometry intersection} and 3D annotation for automatic water level determination without reference gauges for almost every situation regarding running waters. 

Thus, the smartphone application \textit{Open Water Level} that bases on the freely available open source camera framework \textit{Open Camera} \cite{Harman2017}. Open Water Level allows for free stationing water line detection using short hand held time-lapse image sequences (for details please refer to \cite{Kroehnert2017}). 

To interpret these, image measurements must be transformed into object space whereby initial exterior information needs to be provided by smartphone sensors for orientation and positioning. In addition to this, initial intrinsic camera parameters are abstracted by manufacturer specified focal length with fixed pixel skew and image-centred principle point. Distortion is assumed to be zero. In case of successfully determined water line, the user is asked for permission to open a secured FTP connection to access web server. If permission is given, a small file archive is generated and transmitted containing the master image frame from the time lapse sequence, the binarised water line and a XML file that contains additional meta data, which gives information e.g. a unique universal identification number (UUID), users position and orientation, camera specifications or date and time.

Once this package has arrived on the server, first of all the meta data is analysed for user's position, image acquisition date and time to pick a tile of coloured 3D point cloud data captured independently close to river lines for further registration (see section \ref{sec:water_level_gauging_requirements_situation} for more information about gaining reference data). Obviously, the key parameters contain users position in a common global reference frame to choose the right point set. In case of more than one potential match, the acquisition date is included as well to handle possible conflicts in \textit{image-2-geometry intersection} due to seasonal changed appearance of vegetation in near-shore environment or illumination effects due to different daytimes of image and geometry acquisition.

....TBC.... 


\subsubsection{Requirements applying to the sensors}
\label{sec:water_level_gauging_requirements_sensors}
To solve the task of autonomous water level determination on running rivers f.e. emergency cases using \textit{image-2-geometry intersection}, citizen scientists position and orientation must be know. As figured out in \ref{sec:technology}, smartphone sensors accuracies for orientation and location are highly dependent on user's environment. Especially the strong correlation of heading and disturbing magnetic sources may be a issue must be solved specifically related to running rivers where metal railings usually exists. Similar effects can also be noted using high-end IMU systems for instance autonomous car navigation. But the magnetic influences inside cars are almost stable and can be calibrated during the drive (advanced navigation manual). For smartphone orientation, the magnetic strengths attaching the phone may change substantially in short time. A typical scenario would be: a citizen scientist walks along street, taking his phone inside the baggage close to metallic keys. While walking he passes several street lamps, signs, etc. Finally, he arrives at a bridge over a urban river, takes out the phone, looks down to the river and records the time lapse image sequence a few centimetres above a metallic railing. Meanwhile, several cars passing the same bridge. In this simple use case, the magnetic field around the smartphone changes countless times due to several unpredictable disturbances \textcolor{red}{(table_mag_disturb)} \cite{Blum2013}.
\\The heading angle has the highest influence compared to pitch and roll regarding 2D image and 3D object data registration. For this, a so-called synthetic image is rendered from colored 3D reference point clouds using scientist's location and orientation to define a situation-dependent bounding box of points to be projected onto image plane with respect to depth and indentations (see \cite{Boerner2016}). Thereby the heading defines the rotation of the depth direction, as a false angle gives a false viewing direction resulting in a synthetic image that has no similarity or only a little with the time lapse sequence. However, in case of no similarity and thus no possible solution for \textit{image-2-geometry intersection}, simply no water level can be calculated. But in case of slight overlapping, there might be image matches but with very bad distribution that impedes a correct positioning \textcolor{red}{(fig_heading_test)} and may lead to even worse results of false water levels.
\\It is obvious that a second source for destructive results exists: the absolute geo-positioning using smartphones currently installed GNSS receivers. In urban scenes with several shadow effects due to high-rise buildings, errors of several meters in latitude and up to more than 30 meters in height are highly possible where even the weather has impact \cite{Bauer2013, Blum2013, Zandbergen2011}. It is likely that, in the near future, smartphone's GNSS modules will be improved solving lateral accuracies of 50 centimetres \cite{Moore2017}.\\
For now, possible relief might come including other sources for positioning like digital elevation models for simple height correction or invoke map services that allows the user for position refinement. For this, some APIs are already provided by Google \textcolor{red}{(quellen)} but they are rather cost-expensive by extensive accessing. Another upcoming option is including barometers in sensor fusion, altitude can be measured within three meters \cite{Liu2014} but for now, they are not a standard.

\begin{itemize}
\item(table, observation heading during water line detection outside $\rightarrow$ check magnetic strengthens and there changes over short times)
\item (figure/table, sensitivity analysis  $\rightarrow$ heading changed in terms of 10 degrees, what does it make for)
\end{itemize}

\subsubsection{Requirements applying to the scenario}
\label{sec:water_level_gauging_requirements_situation}

\begin{itemize}
\item \textit{online processing and position refinement: need online connection}
\item \textit{image quality for water line detection: influence of image resolution, lightning, ...)}
\item needed reference data! link to extruso project
\end{itemize}

%\begin{itemize}
%\item recap: task to be solved
%\item main requirements for (location- and orientation) sensor accuracy and geometric accuracy
%\item specific requirements to this use case: data availability; illumination; device range to cover
%\item available approach to address the task
%\end{itemize}

\subsection{Field Geology}

\begin{itemize}
\item recap: task to be solved
\item main requirements for (location- and orientation) sensor accuracy and geometric accuracy
\item specific requirements to this use case: data availability; illumination; network inavailability
\item available approach to address the task
\end{itemize}

\subsection{Virtual Field Trips}

\begin{itemize}
\item recap: task to be solved
\item main requirements for (location- and orientation) sensor accuracy and geometric accuracy
\item specific requirements to this use case: data availability; illumination; network inavailability
\item available approach to address the task
\end{itemize}

\subsection{The digital fieldbook}

\begin{itemize}
\item recap: task to be solved
\item main requirements for (location- and orientation) sensor accuracy and geometric accuracy
\item specific requirements to this use case: device range to cover; data integration; no network
\item available approach to address the task
\end{itemize}


\section{Conclusions}
\label{sec:conclusions}

which problems are sufficiently solved ? 
which challenges remain that have already been discussed

\section{Discussion}
\label{sec:discussion}

\begin{itemize}
\item porting existing desktop algorithms on mobile devices [quick and fast]
\item vegetation in scans
\item pre-processing of geodata for mobile use
\end{itemize}

\section*{References}

\bibliography{KroehnertKehl2018_MobileDigitalGeosciences}

\end{document}